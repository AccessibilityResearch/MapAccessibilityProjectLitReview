\documentclass{article}

% Language setting
% Replace `english' with e.g. `spanish' to change the document language
\usepackage[english]{babel}

% Set page size and margins
% Replace `letterpaper' with `a4paper' for UK/EU standard size
\usepackage[letterpaper,top=2cm,bottom=2cm,left=3cm,right=3cm,marginparwidth=1.75cm]{geometry}

% Useful packages
\usepackage{amsmath}
\usepackage{graphicx}
\usepackage[colorlinks=true, allcolors=blue]{hyperref}

% Keywords command
\providecommand{\keywords}[1]
{
  \small	
  \textbf{\textit{Keywords---}} #1
}


\title{Map Accessibility Project (MAP) Literature Review}
\author{Aaron Cherney, Christian Stewart, and Karter Prehn}

\begin{document}
\maketitle

\keywords{Maps, Multi-Modal, Tactile, Audio, Feedback, Visual-Impairment Evaluations, Navigation, Information Gathering, Information Handling}

\section{Introduction}

\par We wanted to explore two areas of interest, maps as tools of information and maps as tools of navigation for the Blind and visually impaired. Through this exploration, our team would be able to decide a direction towards one of those two overarching topics, understanding some of the tools, methods, and design principles that have already been implemented in the pursuit of accessibility and making information available for the Blind and visually impaired. These will influence the creation of our own guidelines for the future development of making maps accessible in a meaningful and effective ways for the Blind and visually impaired, in the context of information retrieval or navigation related to maps.
\newline
\par This review is intended to be a review of available literature, summarizing important findings related to each piece of literature by team members. It is intended to act as a working document, designed for quick reference and guidance, but not a substitution for deeper reference and reading of the original literature. It should also be noted that this literature review likely does not include every important finding that may be relevant from each literature paper reviewed, and due to time constraints, there are many pieces of literature that have not been reviewed. 
\newline
\par It is important to review available literature on existing implementations of tools and suggested guidelines to understand what has been done, and to develop a knowledge base of the subject material, particularly useful in our team’s case to encourage higher level discussion between team members, subject matter expert, and our client. It was through our review that we decided to pursue the category of information as a tool in aiding access of maps for the Blind and visually impaired, with the possibility of exploring implementing an optional layer into Google Maps which can orientate itself at all times in the direction of the user, while presenting relevant information through possible uses of audio and haptic feedback.  
\newline
Our literature review taught several important lessons. These lessons fall into three distinct categories: navigation, information-gathering, and modality. Each of these categories will be reviewed below. 

\section{Literature Discussion and Overview}
This discussion is based off of all the literature reviewed, all reviewed documents can be found in the references section at the bottom of the paper. Much of this discussion will focus on...

\subsection{What problems occur when accessing information from a map}
Based off the literature reviewed there are a few standout problems that appear when accessing information from a map. The main problems include the following: Understanding relational data(where you are relative to known locations)\cite{10.1145/1168987.1169008}, perception of angle/trajectory to a given location\cite{LAHIB201816}, and obtaining relevant data from the map locations such as a location description. Another large problem that was explored in our research was actual application accessibility, many of the largest online map and location services have been tested and shown to be only partially accessible to visually impaired individuals based off Web Content Accessibility Guidelines (WCAG) 2.0\cite{10.1145/2815169.2815171}.

\subsection{What solutions have been explored}
Many different solutions have been explored, Most of these solutions have been under the context of navigation and of these not many have made it past the end user testing, despite showing promising results. An example of this is \cite{10.1145/2468356.2468364}, which describes a waypoint system that's shown to be very promising and makes use of audio and tactile feedback. 

\subsection{How does this work and literature justify our proposed solution}
Many of the problems explored in the literature may be outdated by now or perhaps not relevant anymore. For example, Yahoo maps was shutdown in 2015, making solutions involving it irrelevant. In order to gain more relevant and timely details, we propose surveying a sample population of at least 50 participants. We have chosen to aim for a larger population size because many of the papers reviewed only survey around 15-25 visually impaired participants with their proposed solutions. We feel this was inadequate, and that more feedback will lead to better results.


\section{Lessons Learned From Literature}

Many lessons where learned from the literature review and these lessons fell under a couple different catagories, these catagories are as follows:
\begin{enumerate}
    \item Guidelines for feedback
    \item Guidelines for mobile applications
    \item General guidelines for developing an accessible project
    \item Miscellaneous lessons about some more esoteric solutions
\end{enumerate}
While many lessons where learned and many guidelines where formed in the literature, it is most important to realize that one should still avoid making assumptions and be objective as often as possible.

\subsection{Guidelines For Feedback}

\par When it comes to feedback such as audio and vibrations we can say that there are some general preferences that should be applied to accessible applications. However, specific applications of feedback forms sometimes show specific preferences, for example some participants in \cite{RODRIGUEZSANCHEZ20147210} claimed that constant feedback is preferred while other papers never made this claim. It is also shown in the same paper that a Multi-Modal feedback system (specifically a combination of audio and vibration) is preferred. The latter statement about Multi-Modal feedback has been consistent throughout almost every single paper reviewed. We can see that two statements from the same paper have very different levels of importance when reviewed in context, this is why one must avoid making assumptions whenever possible when dealing with accessibility.
\newline
\par Some more general lessons can be taken from \cite{doi:10.1080/10447318.2017.1279827} where the authors state that accessible mobile applications should deliver appropriate feedback for all visual abilities, meaning that one should accommodate for a variety of different visual impairments (colorblind and partially blind for example). The other lesson taken from this paper is that an application should also deliver appropriate feedback during and after any type of gesture articulation (example, using two fingers to zoom in on a picture). 
\newline
\par Now that we have general guidelines for the feedback forms we explored a paper that studied what kind of physical landmarks blind individuals actually wanted feedback for. In \cite{10.1145/2468356.2468364} the authors explored and tested a wayfinding application and discovered through end-user testing a couple different key guidelines for landmarks and audio queues/feedback that would be used in the context of guiding a user around a physical space outdoors. Lesson 1, ambient sounds should not be overlooked (environmental sounds for example). Lesson 2, intersections and traffic are vitally important when navigating. Lesson 3, landmarks should be immovable and permanent. One last thing that was included in this paper was that a public comment system could be explored so that users could share a spaces "quirks" from a visually impaired perspective.
\newline


\subsection{Guidelines For Mobile Applications/Devices}

\par Only paper that was reviewed gave solid guidelines for the development of mobile applications and devices, however, there are still takeaways to be gleaned. \cite{doi:10.1080/10447318.2017.1279827} 
\newline
\par Studies show that visually impaired people will take more time to configure visual settings in applications when the options are available. Therefore, having such configurations will find ought of use when the application includes them. 
\newline
\par Wearable smart devices have experienced a boom in popularity in recent years, and have potential to provide utility for visually impaired individuals. When creating an application, being mindful of the practicality of such devices can open many opportunities to allow for increased accessibility features.
\newline
\par Unintended touches can pose a plethora of issues when not accounted for by the software. Being able to detect unintended touches, or provide protection against such touches will substantially increase usability of an application. Providing feedback when such touches occur, and having features that allow the user to orient themselves with what is on the screen is useful for keeping the user in the area of the device they wish to be using. 
\newline
\par Currently, gestures are designed to be intuitive for individuals with vision, not for people that are visually impaired. Creating additional gestures, or allowing for customizable/configurable gestures, that are more suitable for visually impaired people will create an application that is easier to use.
\newline
\par Delivering appropriate feedback for all levels of visual ability is imperative for inclusive accessibility. Different gestures, color schemes, vibrations, and sounds may work better for different types of visual impairment. 
\newline
\par Ensuring the user of the application knows what their gesture is doing and when it is complete is vital. A user must be able to know that the intended gesture was registered, and that it has done the intended task. Giving varied feedback for different gestures can tell the user which gesture was input. Providing continuous feedback during some gestures can help the user know that their gesture has not ended prematurely.
\newline
\par Another lesson learned about mobile applications is from \cite{RODRIGUEZSANCHEZ20147210} where the authors state that fixed regions of a device such as the devices corners and buttons are preferred when creating and designing a mobile application. Lastly, \cite{bandyopadhyay2017sound} states very simply that mobile phones offer a low cost means of software accessibility for the visually impaired. The latter statement is a great summation of why we were leaning towards developing some sort of mobile application.




\subsection{General Guidelines For Different Types Of Maps}
When exploring the subject of maps in particular there were many papers that talked about the ability to use 3D printed maps as a tool for the visually impaired. I think it is important to note here that we did not consider going in this direction because our lessons learned showed it to be less than promising as a solution to our problem.
\newline
\cite{10.1007/978-3-319-41267-2_20}
\newline
\par 3D printing a map that combines PLA and conductive materials is often not a viable option due to the resource requirements and time consumption. However, for areas that are not subject to change, and are intended for use by many people, it can be a very effective option. For example, an attraction that has a large volume of people going through it, and does not undergo large-scale changes often, such as a museum or mall, would be an ideal environment for such a map. It is a situation where it is not meant for a single individual to use, but will see much use based on the volume of different individuals that pass through such a place. 3D printing a map for individual use, and for each time said individual wishes to take a trip, is not going to be a reasonable feat for many. 
\cite{10.1145/2207676.2207734}
\newline
\par Braille maps can be useful, and still have several problems. Braille maps are often costly due to their specialized production process. Specially trained individuals, as well as equipment, is needed to produce a braille map. Common, often large scale, maps are more widely produced, however, when a smaller, more detailed map is necessary, it almost always must be custom designed and printed. This process is very involved, and is not instant, in the way that modern mobile applications are able to otherwise provide directions. The level of detail in such braille maps may also be insufficient for proper navigation, depending on how it was designed and created. Another limitation to such maps is they cannot be updated, and the user will not have current information when anything in the environment changes. This poses a challenge in terms of difficulty in navigating, as well as a potentially significant safety risk to any user. 
\newline
The authors built SpaceSense, which uses vibrotactile feedback to provide users information about distance and direction towards a particular destination.
\newline
\par The study shows a positive correlation between the usage of SpaceSense and the ability for participants to maintain spatial relationships between places.


\cite{10.1145/1851600.1851606}
\par The authors created Timbremap, a tool that uses a sonification interface to allow touch-screen mobile devices to provide audio feedback about indoor layouts and geometry.
\newline
\par Their users reported that the tool was effective in conveying, "non-trivial geometry" and helped visually-impaired users to explore indoor layouts.
\newline

\cite{10.1145/3186894}
3d printing a map and adding audio feedback was demonstrated to be a great way to provide information.



\subsection{Fitts's Law}
Fitts' Law states that the amount of time required for a person to move a pointer (e.g., mouse cursor) to a target area is a function of the distance to the target divided by the size of the target. Therefore, it takes longer the farther away the target is and the smaller it is.This is important to make accessible maps and interfaces for visually impaired users.Using Fitts's Law, we can extrapolate how difficult it may be to perform an action on a map, and we can even put a number on it and possibly rank certain interfaces based on difficulty.
\newline
\par However, the biggest issue with this Law is that we tend to fall into the habit of making "assumptions" based on math. Our team then decided to explore many different papers that address this potential problem with Fitts's Law.
Here is what we learned from the following papers:
\newline
First, Fitts' Law can be applied to touchscreens with blind users when the distance and target size are being considered, but not when angle is being considered. 
Then, even a single feedback medium can be useful in this context. (Varying distance and size).
\cite{LAHIB201816}
\newline
Next, the reliability of re-testing Fitts' Law is limited especially when evaluating with impaired users.Finally, when testing for accessibility reasons, multiple testing sessions should be utilized. 
\cite{10.1145/3373625.3416999}

\subsection{Miscellaneous Valuable Lessons}
\cite{10.1145/2815169.2815171} \textbf{Investigating Accessibility on Web-based Maps } 

\par Tool-based evaluation falls short to expert evaluation of Having a web accessibility expert’s opinions generally can correlate to better end user performance. After expert analysis, Google Maps was missing 18 of the 23 accessibility criteria. For instance, Google Maps fails at criteria 1.4.1, as text does not have a luminosity contrast ratio of at least 5:1 compared to its background.
\newline

\cite{10.1007/978-3-319-07440-5_15} \textbf{Usability Evaluation of a Web System for Spatially Oriented Audio Descriptions of Images Addressed to Visually Impaired People } 

\par Something else worth noting is that the screen reader is an important part of a user’s experience with anything they interact with, especially the testing environment. During the testing process, the participants using a screen reader performed tasks significantly faster than those who did not.
\newline

\cite{10.1145/1168987.1169008} \textbf{Using an Audio Interface to Assist Users Who Are Visually Impaired with Steering Tasks }
\par For the presentation of information, on the other hand, graphs that show relational information, as opposed to abstract data, are preferred. When tested side-by-side, graphs with relational information are more effective.
\newline

\cite{watanabe2017evaluation}\textbf{Evaluation of Virtual Tactile Dots on Touchscreens in Map Reading: Perception of Distance and Direction }
\par Tactile dot displays can be an effective way to visualize map information for visually impaired and blind people. Despite this, virtual tactile dot displays are often large, impractical, and overpriced for widespread use. A smartphone, for instance, does not have enough vibration motors within it to accurately produce a virtual tactile dot.
\newline

\cite{thompson2018examination}\textbf{Examination of the Level of Inclusion of Blind Subjects in the Development of Touchscreen Accessibility Technologies }

\par Finally, it is important in the conception of the project to keep the intended users in mind. A project designed and developed for blind individuals is most effective when blind people are involved in the design and development process.



\newpage
\bibliographystyle{alpha}
\bibliography{main.bib}

\end{document}